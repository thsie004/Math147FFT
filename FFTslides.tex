

%%%%%%%%%%%%%%%%%%%%%%%%%%%%%%%%%%% 
%%%%%%%%%%%                                           %%%%%%%%%%%
%%%%%%%%%%%   Various Settings Below   %%%%%%%%%%%
%%%%%%%%%%%                                           %%%%%%%%%%%
%%%%%%%%%%%%%%%%%%%%%%%%%%%%%%%%%%% 



\documentclass{beamer}

\usetheme{bjeldbak}


\usepackage[english]{babel}
% or whatever
\usepackage[makeroom]{cancel}
\usepackage[latin1]{inputenc}
\usepackage{amsmath}
% or whatever

\usepackage{times}
\usepackage[T1]{fontenc}
% Or whatever. Note that the encoding and the font should match. If T1
% does not look nice, try deleting the line with the fontenc.

\usepackage{graphicx}
\definecolor{mygray}{gray}{0.7}
\usepackage{tikz}
\usetikzlibrary{matrix,arrows}
\tikzset{node distance=2cm, auto}


\title[] % (optional, use only with long paper titles)
{The Fast Fourier Transform}

%\subtitle{Your Subtitle}% You don't need this

\author[] % (optional, use only with lots of authors)
{}
% - Give the names in the same order as the appear in the paper.
% - Use the \inst{?} command only if the authors have different
%   affiliation.

\institute[] % (optional, but mostly needed)
{
  \inst{1}%
	
  \and
  \inst{2}%
  
 }
% - Use the \inst command only if there are several affiliations.
% - Keep it simple, no one is interested in your street address.

\date[Mathematics 149C Conference] % (optional, should be abbreviation of conference name)
{Mathematics 149C Conference\\Department of Mathematics - Riverside, California\\June 8th, 2016}
% - Either use conference name or its abbreviation.
% - Not really informative to the audience, more for people (including
%   yourself) who are reading the slides online



\AtBeginSubsection[]
{
  \begin{frame}<beamer>{Outline}
    \tableofcontents[currentsection,currentsubsection]
  \end{frame}
}

\beamerdefaultoverlayspecification{<+->}



%%%%%%%%%%%%%%%%%%%%%%%%%%%%%%%%%%% 
%%%%%%%%%%%                                           %%%%%%%%%%%
%%%%%%%%%%%   Various Settings Above   %%%%%%%%%%%
%%%%%%%%%%%                                           %%%%%%%%%%%
%%%%%%%%%%%%%%%%%%%%%%%%%%%%%%%%%%% 








%%%%%%%%%%%%%%%%%%%%%%%%%%%%%%%% 
%%%%%%%%%%%                                 %%%%%%%%%%%
%%%%%%%%%%%   Begin Document   %%%%%%%%%%%
%%%%%%%%%%%                                 %%%%%%%%%%%
%%%%%%%%%%%%%%%%%%%%%%%%%%%%%%%% 



\begin{document}


%%%%%%%%%%%%%%%%%%%%%%%%%%%%%% 
%%%%%%%%%%%                          %%%%%%%%%%%
%%%%%%%%%%%   New Section   %%%%%%%%%%%
%%%%%%%%%%%                          %%%%%%%%%%%
%%%%%%%%%%%%%%%%%%%%%%%%%%%%%% 




\section[intro]{Introduction} %If the names are too long, you can abbreviate in this way.

\begin{frame}
The Fast Fourier Transform
\begin{itemize}
\item First used by Carl Friedrich Gauss but went unnoticed.
\item James Cooley and John Tukey independently discovered the Cooley-Tukey algorithm in 1965.
\item Improves the $O(N^2)$ running time it takes to perform DFT to $O(N\log_2{N})$ (regarding complex addition and multiplication).
\item
\begin{tabular}{|p{1cm}|p{2cm}|p{2cm}|p{2cm}|}
\hline
&n=1000&n=$10^6$&n=$10^9$\\
\hline
$n^2$&$10^6$&$10^{12}$&$10^{18}$\\
\hline
$n\log_2{n}$&\textasciitilde$10^4$&\textasciitilde$2\cdot10^7$&\textasciitilde$3\cdot10^{10}$\\
\hline
\end{tabular}
\end{itemize}

\end{frame}

\section[Polynomials]{Operations on Polynomials}

\begin{frame}
A degree $N-1$ polynomial is given by $P(x) = a_0+a_1x+a_2x^2+\cdots+a_{N-1}x^{N-1}$.\\
\bigskip
\large{Other representations:}
\begin{itemize}
\item{\textbf{Coefficient vector:}} A vector $(a_0,a_1,\cdots,a_{N-1})$ in a space of monomials.
\item{\textbf{Roots:}} Given roots of a polynomial, we can write $P(x) = k\prod_{n=0}^{N-1}(x-r_n)$ for some constant $k$.
\item{\textbf{Samples:}} We can also take $N$ distinct pairs of $(x_n,y_n=P(x_n))$ to uniquely determine the polynomial $P(x)$ by Lagrange polynomial interpolation.
\end{itemize}
\end{frame}

\begin{frame}
There are three main things we want to do with polynomials, namely:
\begin{itemize}
\item{\textbf{Evaluation:}} Evalutae the polynomial at a point in its domain.
\item{\textbf{Addition:}} Add two polynomials together.
\item{\textbf{Multiplication:}} Multiply two polynomials together. 
Note that polynomial multiplication in the coefficient vector form actually corresponds to the finite discrete convolution of two vectors in their inner product space. 
\end{itemize}
\end{frame}

\begin{frame}

\begin{tabular}{|p{2cm}||p{2cm}|p{2cm}|p{2cm}|}
\hline
\multicolumn{4}{|c|}{Running-time for each operation on degree $N-1$ polynomials}\\
\hline
&Coeffcients&Roots&Samples\\
\hline\hline
Evaluation&$O(N)$&$O(N)$&$O(N^2)$\\
\hline
Addition&$O(N)$&N/A&$O(N)$\\
\hline
Multiplication&$O(N^2)$&$O(N)$&$O(N)$\\
\hline
\end{tabular}

\bigskip
Objective: convert coefficients to samples efficiently as to avoid $O(N^2)$ running-time.
\end{frame}

\section[Transformation 1.0]{Transformation 1.0}

\begin{frame}
	We can sample $P(x) = a_0+a_1x+a_2x^2+\cdots+a_{N-1}x^{N-1}$ by doing \\
	\bigskip
	\begin{centering}{
		$\begin{pmatrix}
			1 & x_0 & x_0^2 & \cdots & x_0^{N-1}\\ 
			1 & x_1 & x_1^2& \cdots & x_1^{N-1}\\ 
			\vdots & \vdots & \vdots & \ddots & \vdots\\ 
			1 & x_{N-1}  & x_{N-1}^{2}& \cdots & x_{N-1}^{N-1}
		\end{pmatrix}
		\begin{pmatrix}
			a_0\\ 
			a_1\\ 
			\vdots\\ 
			a_{N-1}
		\end{pmatrix}
		=
		\begin{pmatrix}
			y_0\\ 
			y_1\\ 
			\vdots\\ 
			y_{N-1}
		\end{pmatrix}$.
	}

\end{centering}

\bigskip
This calculates $N$ samples at $N$ distinct points in the set $X = \{x_0,\cdots,x_{N-1}\}$ with a running-time of $O(N^2)$...
\end{frame}

\begin{frame}
\onslide<+->{Divide and conquer! (this is where we assume $N$ is a power of 2)}

\bigskip
\onslide<+->{
To evaluate our polynomial $P(x)$ at the point $x$, we can define the following:}
\begin{itemize}
\item{$P_{\text{even}}(x)=\displaystyle \sum_{k=0}^{[(N-1)/2]} a_{2k}x^k$}
\item{$P_{\text{odd}}(x)=\displaystyle \sum_{k=0}^{[(N-1)/2]} a_{2k+1}x^k$}
\item{$P(x) = P_{\text{even}}(x^2)+x\cdot P_{\text{odd}}(x^2)$}
\end{itemize}
\onslide<+->{Do this for every $x$ we want to sample... $N$ of them.}
\end{frame}

\begin{frame}
\onslide<+->{
Let $T$ measure the running-time of calculating $P(x)$ for every $x$ in $X$, then we have that
\begin{eqnarray*}
T(N,|X|) &=& 2T(\frac{N}{2}, |X|) + O(|X|)\\
&=& 2\left[2T(\frac{N}{4}, |X|)+O(|X|)\right]+O(|X|)\\
&\vdots&\\
&=& 2^{\log_2{N}}T(1,\mathbf{|X|})+\sum_{l=0}^{\log_2{N}}2^lO(|X|)\\
&=& \mathbf{N^2}+O(|X|\log_2{N}).
\end{eqnarray*}
}
\onslide<+->{$O(N^2)$ again!}

\onslide<+->{Root cause? This algorithm chokes when it has to evaluate monomials \textbf{for every element in the set}.}
\end{frame}

\section[Roots of Unity]{Roots of Unity}

\begin{frame}
\onslide<+->{What if there is a set that shrinks in size whenever we square each element?} 
\begin{eqnarray*}
\onslide<+->{\{1\} &=& \{e^{2\pi i\frac{1}{1}}\}, N=1}\\
\onslide<+->{\{-1,1\} &=& \{e^{2\pi i\frac{1}{2}}, e^{2\pi i\frac{2}{2}}\}, N=2}\\
\onslide<+->{\{i,-1,-i,1\} &=& \{e^{2\pi i\frac{1}{4}},e^{2\pi i\frac{2}{4}},e^{2\pi i\frac{3}{4}}, e^{2\pi i\frac{4}{4}}\}, N=4\\
&\vdots&}
\end{eqnarray*}
\onslide<+->{Eventually end up with \textbf{the Nth roots of unity} $\{e^{2\pi i\frac{k}{N}}\}_{k=0}^{N-1}$.}

\onslide<+->{Key property: any such root computes to 1 when raised to the Nth power.}
\end{frame}

\section[Transformation 2.0]{Transformation 2.0}

\begin{frame}
\onslide<+->{Why not sample our polynomial at the Nth roots of unity?}
\onslide<+->{
\begin{eqnarray*}
	T(N) &=& 2T(\frac{N}{2}) + O(N)\\
	&=& 2\left[2T(\frac{N}{4})+O(\frac{N}{2})\right]+O(N)\\
	&\vdots&\\
	&=& 2^{\log_2{N}}{T(1)}+\sum_{l=0}^{\log_2{N}}2^lO(\frac{N}{2^l})\\
	&=& N+O(N\log_2{N})
\end{eqnarray*}
}
\onslide<+->{Hello, $O(N\log_2{N})$!}

\onslide<+->{But what does converting polynomial forms have to do with FFT?}
\end{frame}

\section[The DFT Matrix]{The DFT Matrix}

\begin{frame}
\onslide<+->{Recall our choice to sample the polynomial on $\{e^{2\pi i\frac{k}{N}}\}_{k=0}^{N-1}$}
\onslide<+->{, this gives us the following transformation:

\bigskip
\begin{centering}{
		$\begin{pmatrix}
		1 & (e^{2\pi i\frac{0}{N}})^1 & (e^{2\pi i\frac{0}{N}})^2 & \cdots & (e^{2\pi i\frac{0}{N}})^{N-1}\\ 
		1 & (e^{2\pi i\frac{1}{N}})^1 & (e^{2\pi i\frac{1}{N}})^1 & \cdots & (e^{2\pi i\frac{1}{N}})^{N-1}\\ 
		\vdots & \vdots & \vdots & \ddots & \vdots\\ 
		1 & (e^{2\pi i\frac{N-1}{N}})^1  & (e^{2\pi i\frac{N-1}{N}})^{2}& \cdots & (e^{2\pi i\frac{N-1}{N}})^{N-1}
		\end{pmatrix}
		\begin{pmatrix}
		a_0\\ 
		a_1\\ 
		\vdots\\ 
		a_{N-1}
		\end{pmatrix}$.\\
}
\end{centering}
}
\bigskip
\onslide<+->{Now recall again the discrete Fourier coefficients $\displaystyle A_n = \frac{1}{N}\sum_{k=0}^{N-1} F(k)(e^{-2\pi i\frac{n}{N}})^{k}$ for $n$ running from $0$ to $N-1$.} 

\onslide<+->{Replace $a_k$ with $\frac{1}{N}F(k)$ and take the conjugate of the DFT matrix then we have Cooley and Tukey's radix-2 FFT.}

\end{frame}

\section[Bringing it Back]{Bringing it Back}
\begin{frame}
\onslide<+->{After multiplication/convolution, we need to convert samples back into coefficients.}

\onslide<+->{Luckily, this is easy to do!}

\onslide<+->{
	Let V be the DFT matrix as described earlier, then
	\begin{equation*}
		V^{-1} = \frac{1}{N}\bar{V},
	\end{equation*}
	which can be proven by showing that
	\begin{equation*}
		V\bar{V} = NI.
	\end{equation*}
}
\onslide<+->{
	Then $\vec{a} = V^{-1}\vec{y}$.
}
\end{frame}

%%%%%%%%%%%%%%%%%%%%%%%%%%%%%%% 
%%%%%%%%%%%                              %%%%%%%%%%%
%%%%%%%%%%%   End Document   %%%%%%%%%%%
%%%%%%%%%%%                              %%%%%%%%%%%
%%%%%%%%%%%%%%%%%%%%%%%%%%%%%%% 

\end{document}







